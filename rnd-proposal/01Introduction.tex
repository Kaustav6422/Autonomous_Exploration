\section{Introduction}
\label{sec:intro}
Maps are necessary part of mobile robots, they require it for navigation through environment. To build
maps robots can explore there surroundings using there perception sensors(like laser sensors,
Ultrasonic sensors and camera). Simplest way to explore an area using mobile robot is to give movement
commands to robot(tele-operation) and store the occupancy information of small patches of the environment,
this kind of map is called occupancy grid maps. Exploration through tele-operation is non-ideal in long
distance exploration 'like mars planetary exploration' due to communication delay, in this scenario
autonomous exploration is a use-full tool to employ.
\par
Formally \textit{Occupancy Grid Maps} represent the occupation probability of each zone of the
environment within a grid \cite{Juliae2012}.
\par
An exploration plan is simply a series of nodes of a graph connected by edges,
In which each node represents a position(which is a pair of Cartesian-Coordinates in a 2D map)
to be visited by the robot. If the robot visits all nodes in the given sequence,
it has completed exploration according to given plan(for example look at Figure 1).
On the low-level these coordinates can be translated to the series of commands which are to be
executed by drive of the robot, which can vary from robot to robot.
\par

\picHereWidth{map_example.png}{Example of an in-door environment \cite{Moorehead2001} \textit{left:} original structure of the real in-door environment with x representing the starting position of the robot. \textit{right:} explored map: with crosses representing the visited positions by the robot.}{Figure 1}{scale=0.7}

\subsection{Prior Work}
\label{sec:priorwork}
