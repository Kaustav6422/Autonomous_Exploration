\section{Introduction}
\label{sec:intro}
Maps are necessary part of mobile robots, they require it for navigation through environment. To
build maps robots can explore there surroundings using there perception sensors(like laser sensors,
Ultrasonic sensors and camera). Simplest way to explore an area using mobile robot is to give
movement commands to robot(tele-operation) and store the occupancy information of small patches of
the environment, this kind of map is called occupancy grid maps. \par

Exploration through tele-operation is non-ideal in long distance exploration 'like mars planetary
exploration' due to communication delay, in this scenario autonomous exploration is a use-full tool
to employ.In domestic environments like offices and homes mapping is done to locate objects present
in the room, one major constraint is area coverage. For, maximum coverage the autonomous explorer
must be fast and explore every corner of the environment. Conversely, for an industrial environment
the explorer will have to be accurate, discover every important feature of the map with calibrated
sensors. \par

Aforementioned constraints are satisfied by different exploration strategies. An \textit{exploration
strategy} is a general method of devising an exploration plan. It is not necessary that a complete
plan is generated before execution, usually explorers follow a analysis-execution loop , in which
they analyze based on sensory information which direction to go and then execute motion according to
the direction. A number of exploration strategies are mention in next section with there advantages.
For simplicity we are concentrating on single robot-exploration, not all but some of the strategies
are scale-able to multi robot teams by adding knowledge sharing amongst robots. \par

An exploration plan is simply a series of nodes of a graph connected by edges, In which each node
represents a position(which is a pair of Cartesian-Coordinates in a 2D map) to be visited by the
robot. If the robot visits all nodes in the given sequence, it has completed exploration according
to given plan(for example look at Figure 1). On the low-level these coordinates can be translated to
the series of commands which are to be executed by drive of the robot, which can vary from robot to
robot. \par

\picHereWidth{map_example.png}{Example of an in-door environment \cite{Moorehead2001} \textit{left:}
original structure of the real in-door environment with x representing the starting position of the
robot. \textit{right:} explored map: with crosses representing the visited positions by the robot.}
{Figure 1}{scale=0.7}

\subsection{Prior Work}
\label{sec:priorwork}
The simplest and most intuitive way to explore is presented by \cite{Yamauchi1997}, to start by
identifying frontiers in the evidence(occupancy) grid and then going to the nearest frontier.
According to Yamauchi et. al. \cite{Yamauchi1997} \textit{Frontiers} are regions on the boundary
between open space and unexplored space. The research also offers a good solution to specular
reflection problem of sonar range finder.\par

A distinct approach of Next best View to explore is presented by \cite{Gonzalez-Banos2002}.
Computing a sequence of sensing positions based on data acquired at previous locations is referred
to as the next-best-view(NBV) problem. The presented algorithm maximizes the information gain over
the map and thus improving the map quality, which is a major metric of exploration performance
according to \cite{Yan2015}.\par

A similar approach of NBV was employed with inverse sensor models by \cite{Grabowski2003}, in which
the next best view is chosen by inverse sensor model. As opposed to Gonzalez et. al. where next best
view is the point of maximum exposure of the robot. Authors also suggest an indirect solution to
specular reflection problem of the ultrasonic sensors.\par

Newman et. al. in \cite{Newman2003} devised a feature based autonomous exploration technique, which
integrates features to reduce the uncertainty in the SLAM. Features are select by novelty check
which in-turn makes the robot to explore unexplored areas. This algorithm is an example of
'Integrated' algorithm suggested by \cite{Juliae2012}.\par

A technique developed for the multi-robot system by \cite{Burgard2005}, which also works for single robot does
cost of analysis of exploring a frontier. It is interesting to know that cost is not computed over
the straight line distance between robot and frontier but rather cells required to traverse till
frontier. It also computes the utility of the frontier and choses the frontier with maximum utility
and least cost, for multi-agent system the utility of a frontier is reduced every time a robot is
assigned to it. They also consider limited communication distance scenario in a robot team.\par

Stachniss et. al.\cite{Stachniss2005} consider information gain in exploring a frontier. The concept
is little similar to Burgard et. al., but the utility of a frontier is the amount of information
gain of the Rao-Blackwellized particle filter(RBPF). They also consider the utility of the path
taken to the frontier, which is affected by the gain of the path traversed. For example gain of the
path is greater which causes loop-closure in the SLAM process. \par

All of the above mentioned algorithms can be evaluated with the help of the benchmarking metrics
given by Yan et. al.\cite{Yan2015}. These metrics can be considered according to application for
which the robots are needed, which is also the focus of Yan et. al. Most of the algorithm discussed
above are evaluated by Juliae et. al.\cite{Juliae2012}. They considered some metrics given by
Yan et. al. and further discuses other metrics for multi robot teams.
